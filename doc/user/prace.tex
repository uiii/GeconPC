%%% Hlavní soubor. Zde se definují základní parametry a odkazuje se na ostatní části. %%%

%% Verze pro jednostranný tisk:
% Okraje: levý 40mm, pravý 25mm, horní a dolní 25mm
% (ale pozor, LaTeX si sám přidává 1in)
\documentclass[12pt,a4paper]{article}
%\setlength\textwidth{145mm}
\setlength\textheight{247mm}
%\setlength\oddsidemargin{15mm}
%\setlength\evensidemargin{15mm}
\setlength\topmargin{0mm}
\setlength\headsep{0mm}
\setlength\headheight{0mm}
% \openright zařídí, aby následující text začínal na pravé straně knihy
\let\openright=\clearpage

%% Pokud tiskneme oboustranně:
% \documentclass[12pt,a4paper,twoside,openright]{report}
% \setlength\textwidth{145mm}
% \setlength\textheight{247mm}
% \setlength\oddsidemargin{15mm}
% \setlength\evensidemargin{0mm}
% \setlength\topmargin{0mm}
% \setlength\headsep{0mm}
% \setlength\headheight{0mm}
% \let\openright=\cleardoublepage

%% Pokud používáte csLaTeX (doporučeno):
\usepackage{czech}
%% Pokud nikoliv:
%\usepackage[czech]{babel}
%\usepackage[T1]{fontenc}

%% Použité kódování znaků: obvykle latin2, cp1250 nebo utf8:
\usepackage[utf8]{inputenc}

%% Ostatní balíčky
\usepackage{graphicx}
\usepackage{tikz}
\usepackage{tikz-3dplot}

\usetikzlibrary{patterns}
\usetikzlibrary{decorations.pathreplacing}

\usepackage{mathtools}
\usepackage{listings}

\usepackage{amsthm}
\usepackage{amssymb}

\usepackage{needspace}

\newenvironment{priklad}{%
    \par%
    \vspace{1em}%
    \leftskip=2\parindent
    \rightskip=2\parindent
    \small
    \needspace{3\baselineskip}
    \begin{sloppypar}%
    \noindent\textit{\textbf{Příklad.}}%
}{%
    \end{sloppypar}%
    \vspace{1em}%
}

\usepackage{array}
\newcolumntype{t}{>{\tt}}

\usepackage[disable, textsize=tiny]{todonotes}
\usepackage[bottom, hang]{footmisc}

\usepackage{caption}
\usepackage[noend]{algpseudocode}
\usepackage{algorithm}
\floatname{algorithm}{Algoritmus}

\newcommand{\var}[1]{\textit{#1}}
\newcommand{\func}[2]{\mbox{\textsc{#1}}\ifx&#2&\else(#2)\fi}
\newcommand{\keyword}[1]{\textbf{#1}}
\newcommand{\NULL}{{\small NULL }}

\newcommand{\class}[1]{\texttt{#1}}
\newcommand{\method}[2]{\texttt{\mbox{#1}\ifx&#2&()\else(#2)\fi}}
\newcommand{\function}[2]{\texttt{\mbox{#1}\ifx&#2&()\else(#2)\fi}}

\usepackage{alltt}

%\usepackage{showframe}

%% Balíček hyperref, kterým jdou vyrábět klikací odkazy v PDF,
%% ale hlavně ho používáme k uložení metadat do PDF (včetně obsahu).
%% POZOR, nezapomeňte vyplnit jméno práce a autora.
%\usepackage[ps2pdf,unicode]{hyperref}   % Musí být za všemi ostatními balíčky
\usepackage[unicode]{hyperref}   % Musí být za všemi ostatními balíčky
\hypersetup{
    colorlinks,%
    citecolor=black,%
    filecolor=black,%
    linkcolor=black,%
    urlcolor=black
}
\hypersetup{pdftitle=Ovládání počítače pomocí jednobarevných objektů snímaných
kamerou - Uživatelská dokumentace}
\hypersetup{pdfauthor=Richard Jedlička}

\usepackage[all]{hypcap}

\newenvironment{itemize_compact}{\begin{list}{$\bullet$}{
    \setlength{\parsep}{0mm}
    \setlength{\itemsep}{0mm}
}}{\end{list}}%

\newcommand{\sectionline}{%
  %\nointerlineskip 
  %\vspace{\baselineskip}%
  \begin{center}
  \noindent\rule{\linewidth}{.5pt}%
  \end{center}
  %\par\nointerlineskip
  %\par% \vspace{\baselineskip}
}

\usepackage{glossaries}
\usepackage{xkeyval}
\usepackage{xstring}

\makeatletter

\newcommand{\termdef}[2]{%
    \csname term@#1@#2\endcsname%
}

\newcommand{\newterm}[2]{%
    \define@cmdkeys{term}[term@#1@]{%
        description,plural,sort,see,%
        1pad,2pad,3pad,4pad,5pad,6pad,7pad%
    }[none]
    \setkeys{term}{#2}
    \newglossaryentry{#1}{%
        name={\termdef{#1}{1pad}},%
        description={\termdef{#1}{description}},%
        plural={\termdef{#1}{plural}},%
        sort={\termdef{#1}{sort}},%
        see={\termdef{#1}{see}},%
        first={\itshape\termdef{#1}{1pad}}%
    }
}

\newcommand{\term}[2][1pad]{%
    \IfEq{#1}{1pad}{%
        \gls{#2}%
    }{%
        \glslink{#2}{\termdef{#2}{#1}}%
    }%
}

\newcommand{\Term}[2][1pad]{%
    \IfEq{#1}{1pad}{%
        \Gls{#2}%
    }{%
        \Glslink{#2}{\termdef{#2}{#1}}%
    }%
}

\makeatother

%%% Drobné úpravy stylu

% Tato makra přesvědčují mírně ošklivým trikem LaTeX, aby hlavičky kapitol
% sázel příčetněji a nevynechával nad nimi spoustu místa. Směle ignorujte.
%\makeatletter
%\def\@makechapterhead#1{
  %{\parindent \z@ \raggedright \normalfont
   %\Huge\bfseries \thechapter. #1
   %\par\nobreak
   %\vskip 20\p@
%}}
%\def\@makeschapterhead#1{
  %{\parindent \z@ \raggedright \normalfont
   %\Huge\bfseries #1
   %\par\nobreak
   %\vskip 20\p@
%}}
%\makeatother

% Toto makro definuje kapitolu, která není očíslovaná, ale je uvedena v obsahu.
%\def\chapwithtoc#1{
%\chapter*{#1}
%\addcontentsline{toc}{chapter}{#1}
%}

\linespread{1.3}

\clubpenalty 10000
\widowpenalty 10000

\usepackage{encxvlna}
\begin{document}

% Trochu volnější nastavení dělení slov, než je default.
%\lefthyphenmin=2
%\righthyphenmin=2

%%% Titulní strana práce

\pagestyle{empty}
\begin{center}

\large

Univerzita Karlova v Praze

\medskip

Matematicko-fyzikální fakulta

%\centerline{\mbox{\includegraphics[width=60mm]{img/logo.pdf}}}

\vfill
%\vspace{5mm}

{\large Richard Jedlička}

\vspace{30mm}

% Název práce přesně podle zadání
{\LARGE\bfseries Ovládání počítače pomocí jednobarevných objektů snímaných kamerou}

\vfill

{\Huge Uživatelská dokumentace}
% Název katedry nebo ústavu, kde byla práce oficiálně zadána
% (dle Organizační struktury MFF UK)

\vfill

% Zde doplňte rok
2012

\end{center}

\newpage

%%% Následuje vevázaný list -- kopie podepsaného "Zadání bakalářské práce".
%%% Toto zadání NENÍ součástí elektronické verze práce, nescanovat.

%%% Na tomto místě mohou být napsána případná poděkování (vedoucímu práce,
%%% konzultantovi, tomu, kdo zapůjčil software, literaturu apod.)

\openright

\newpage

%%% Strana s automaticky generovaným obsahem bakalářské práce. U matematických
%%% prací je přípustné, aby seznam tabulek a zkratek, existují-li, byl umístěn
%%% na začátku práce, místo na jejím konci.

\openright
\pagestyle{plain}
\setcounter{page}{1}
\tableofcontents

\thispagestyle{empty}

%%% Jednotlivé kapitoly práce jsou pro přehlednost uloženy v samostatných souborech

\section{Systémové požadavky}
Aplikace ke svému běhu potřebuje počítač nainstalovaným operačním systémem
\textbf{GNU/Linux}. Dále vyžaduje funkční \textbf{webovou kameru}. 

\section{Softwarové závislosti}

\section{Sestavení}
\section{Instalace}


%%% Tabulky v bakalářské práci, existují-li.
%\chapter*{Seznam tabulek}
%\addcontentsline{toc}{chapter}{Seznam tabulek}

%%% Použité zkratky v bakalářské práci, existují-li, včetně jejich vysvětlení.
%\chapter*{Seznam použitých zkratek}
%\addcontentsline{toc}{chapter}{Seznam použitých zkratek}

%%% Přílohy k bakalářské práci, existují-li (různé dodatky jako výpisy programů,
%%% diagramy apod.). Každá příloha musí být alespoň jednou odkazována z vlastního
%%% textu práce. Přílohy se číslují.
%\chapter*{Přílohy}
%\addcontentsline{toc}{chapter}{Přílohy}

\openright
\end{document}
